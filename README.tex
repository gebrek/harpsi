% Created 2016-05-16 Mon 17:46
\documentclass[11pt]{article}
\usepackage[utf8]{inputenc}
\usepackage[T1]{fontenc}
\usepackage{fixltx2e}
\usepackage{graphicx}
\usepackage{longtable}
\usepackage{float}
\usepackage{wrapfig}
\usepackage{rotating}
\usepackage[normalem]{ulem}
\usepackage{amsmath}
\usepackage{textcomp}
\usepackage{marvosym}
\usepackage{wasysym}
\usepackage{amssymb}
\usepackage{hyperref}
\tolerance=1000
\author{Jacob Sonnenberg}
\date{\today}
\title{Building a DSL for Music Composition}
\hypersetup{
  pdfkeywords={},
  pdfsubject={},
  pdfcreator={Emacs 24.5.1 (Org mode 8.2.10)}}
\begin{document}

\maketitle
\tableofcontents

\section{The Language}
\label{sec-1}
The language is made up of two major parts, macros and
note-strings. Macros provide some convenience facility for writing
in Elixir while the note-strings are bit-strings which are parsed to
generate the data.
\subsection{Note-Strings in the Language}
\label{sec-1-1}
In Harpsi the actual playable data is decided primarily by
note-strings. The note-strings are parsed into chords and notes and
act as below staff level. Transformation from raw string to
playable data structure is performed by the \url{lib/parser.ex}. The
backbone of the \verb~Parser~ is \verb~process_word~ which serves as a
dispatch to the various rules defining how the notes are
constructed, each case of the cond serve as the individual rules
for note construction. The rules of course can be any elixir
expression. Regular expressions serve as the rules for how notes
are constructed. 

The `bottom' of the language is \verb~process_note~, every word must
decompose into single letters naming the notes A through G, or a
rest. The language for note-strings is essentially a series of
manipulations on \verb~%Note~, \verb~%Chord~, and \verb~%Rest~ structs, and so exists in the
small domain that currently allows. In the future this domain could
be expanded by changing a finite number of (albeit yet
undocumented) functions, and adding more fields to the two structs,
and enhancing the engine. But it serves as a simple basis for my
purposes that permits some limited amount of expression.

As for rules, a good example is \verb~process_doctave~:
\begin{verbatim}
def process_doctave(word, opts) do
  cap = Regex.named_captures(~r/^(?<up_or_down>[<>])(?<word>.*)/, word)
  process_word(cap["word"], Map.put(opts, :octave,
	opts.octave + (case cap["up_or_down"] do
			 "<" -> -1
			 ">" -> 1
		       end)))
end
\end{verbatim}
This one is changes the the octave by a shift of 

Each rule should work on a certain foundation that could
potentially be better enforced as macro. Implementing it is an
exercise for the reader, but I'll note patterns that exist. The
\verb~word~ arg is the individual note-string note in a staff. This is
dissected by the rules via regex into the three basic structs,
\verb~%Note~ and \verb~%Chord~, and \verb~%Rest~. On the other hand \verb~opts~ is
concerned with the rest of the data associated with the notes.

Most rules are built like
\begin{verbatim}
def process_<case>(word, opts) do
  cap = Regex.named_captures(<regex recognizing case>, word)
  process_word(<transformation on cap["word"]>,
    <transformations of the structs>)
end
\end{verbatim}
with a little imagination, one could construct a macro for defining
these rules in a structured way by templating the above snippet and
adding an entry to a list of callback functions invoked by
\verb~process_word~. More generally, the rules should have one of three
things in the tail place of the function. That is,
\begin{enumerate}
\item A recursive call back to \verb~process_word~ with modified parameters
\item A \verb~Note~ struct
\item A \verb~Rest~ struct
\end{enumerate}
Chord not included in the list because they're only a structure
requiring constituent notes, of course.
\subsection{Macros in the Language}
\label{sec-1-2}
All the macros exist in in \url{lib/lang.ex}. The foremost actor is
the \verb~piece~ macro which reflects the \verb~%Piece~ struct. When writing
in Harpsi the \verb~piece~ macro provides a manipulable environment for
writing \verb~Staff~'s of music and building the whole structure of the
playable \verb~%Piece~.

As stated there are two variable dimensions, the `buffer' of music
and the `environment' the notes are created in. In \verb~Lang~ you'll
find I use two agents to model this behavior in a unhygienic way,
requiring a set of functions to handle an ad-hoc, unspecified
behaviors for constructing the buffer and maintaining the
environment. Agents are "simple abstractions around state", some
shared state is kept in it so the state is accessible at different
points in macro expansion

The buffer agent simply accumulates the musical structure and
returns a list of \verb~Staff~'s, and the environment agent tracks the
state of the environment as a stack. Management of the environment
is especially straight forward. The environment is initialized with
the \verb~start_env~ function.
\begin{verbatim}
def start_env(), do: Agent.start_link(fn -> 
  [%{bpm: 120, octave: 4, type: 4}] end)
\end{verbatim}
This starts an agent with an initial element in the stack which
serves as the `default' environment. The environment is maintained
with a set of three functions:
\begin{verbatim}
def push_env(env, attr_map) do
  new = Map.merge(get_env(env), attr_map)
  Agent.update(env, &[new | &1])
end

def get_env(env), do: Agent.get(env, &(&1)) |> hd

def pop_env(env), do: Agent.update(env, &tl/1)
\end{verbatim}
The functional requirements are minimal and the behavior
is pretty intuitive. It's a simple stack that implements push, pop,
and peek. In this way the macro 
% Emacs 24.5.1 (Org mode 8.2.10)
\end{document}
